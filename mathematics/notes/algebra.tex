\documentclass[12pt]{article}
\usepackage{amsmath, amssymb, amsthm}
\usepackage{geometry}
\usepackage{enumerate}

% Adjust the page margins if needed
\geometry{margin=1in}

% Define theorem environments
\newtheorem{theorem}{Theorem}[section]
\newtheorem{lemma}[theorem]{Lemma}
\newtheorem{corollary}[theorem]{Corollary}
\newtheorem{definition}[theorem]{Definition}
\newtheorem{example}[theorem]{Example}
\newtheorem{remark}[theorem]{Remark}

% Define shorthand commands
\newcommand{\RR}{\mathbb{R}}
\newcommand{\ZZ}{\mathbb{Z}}
\newcommand{\NN}{\mathbb{N}}
\newcommand{\QQ}{\mathbb{Q}}
\newcommand{\abs}[1]{\lvert #1 \rvert}
\newcommand{\norm}[1]{\lVert #1 \rVert}
\newcommand{\inner}[2]{\langle #1, #2 \rangle}

\title{Math Notes}
\author{Your Name}
\date{\today}

\begin{document}

\maketitle

\section{Notes}

\begin{itemize}
	\item algebra is a lot like arithmetic, it follows the rules of arithmetic and uses the same four main operations that arithmetic is built on
	\item the difference between algebra and arithmetic is that algebra introduce the element of the unknown
	\item algebraic equation is a mathematical statement that 2 things are equal
	\item one of the main goals of algebra is solving equations
	\item solving equations is to figure out what the unknown values in equations are
	\item multiplication in algebra is the defualt operation which means that multiplication is implied
\end{itemize}


\section{Introduction}

Your introductory text goes here.

\section{Theorems and Proofs}

\begin{theorem}
	This is a sample theorem.
\end{theorem}

\begin{proof}
	This is a sample proof.
\end{proof}

\section{Definitions and Examples}

\begin{definition}
	A Term is a mathematical expression that are made of two different parts (a number part and a variable part), in a term the number part and the variable part is multiplied together, the number part is called the coeffient.

	it's conventional to write the number part first then the variable part.

	a term is $6x$

	a constant term is $6$
\end{definition}

\begin{definition}
	A Polynomial is a series of terms that are joined together by addition or subtraction.

	Specific names for Polynomials are:-

	\begin{itemize}
		\item monomial: 1 term
		\item binomial: 2 terms
		\item trinomial: 3 terms
		\item polynomial: more than 3 terms
	\end{itemize}

	it's also common to use the word "polynomial" for 2 or 3 terms

\end{definition}

\begin{definition}
	The degree of a term is the power of its variable part

	The degree of a polynomial is the highest power of its variable parts
	\begin{itemize}
		\item monomial: 1 term
		\item binomial: 2 terms
		\item trinomial: 3 terms
		\item polynomial: more than 3 terms
	\end{itemize}

	it's also common to use the word "polynomial" for 2 or 3 terms

\end{definition}

\begin{example}

\end{example}

\section{Conclusion}

Your concluding thoughts go here.

\end{document}
